\documentclass[]{article}
\usepackage{lmodern}
\usepackage{amssymb,amsmath}
\usepackage{ifxetex,ifluatex}
\usepackage{fixltx2e} % provides \textsubscript
\ifnum 0\ifxetex 1\fi\ifluatex 1\fi=0 % if pdftex
  \usepackage[T1]{fontenc}
  \usepackage[utf8]{inputenc}
\else % if luatex or xelatex
  \ifxetex
    \usepackage{mathspec}
  \else
    \usepackage{fontspec}
  \fi
  \defaultfontfeatures{Ligatures=TeX,Scale=MatchLowercase}
\fi
% use upquote if available, for straight quotes in verbatim environments
\IfFileExists{upquote.sty}{\usepackage{upquote}}{}
% use microtype if available
\IfFileExists{microtype.sty}{%
\usepackage{microtype}
\UseMicrotypeSet[protrusion]{basicmath} % disable protrusion for tt fonts
}{}
\usepackage[margin=1in]{geometry}
\usepackage{hyperref}
\hypersetup{unicode=true,
            pdftitle={Tables for Open Computer Science},
            pdfauthor={Franziska Hirt},
            pdfborder={0 0 0},
            breaklinks=true}
\urlstyle{same}  % don't use monospace font for urls
\usepackage{color}
\usepackage{fancyvrb}
\newcommand{\VerbBar}{|}
\newcommand{\VERB}{\Verb[commandchars=\\\{\}]}
\DefineVerbatimEnvironment{Highlighting}{Verbatim}{commandchars=\\\{\}}
% Add ',fontsize=\small' for more characters per line
\usepackage{framed}
\definecolor{shadecolor}{RGB}{248,248,248}
\newenvironment{Shaded}{\begin{snugshade}}{\end{snugshade}}
\newcommand{\KeywordTok}[1]{\textcolor[rgb]{0.13,0.29,0.53}{\textbf{#1}}}
\newcommand{\DataTypeTok}[1]{\textcolor[rgb]{0.13,0.29,0.53}{#1}}
\newcommand{\DecValTok}[1]{\textcolor[rgb]{0.00,0.00,0.81}{#1}}
\newcommand{\BaseNTok}[1]{\textcolor[rgb]{0.00,0.00,0.81}{#1}}
\newcommand{\FloatTok}[1]{\textcolor[rgb]{0.00,0.00,0.81}{#1}}
\newcommand{\ConstantTok}[1]{\textcolor[rgb]{0.00,0.00,0.00}{#1}}
\newcommand{\CharTok}[1]{\textcolor[rgb]{0.31,0.60,0.02}{#1}}
\newcommand{\SpecialCharTok}[1]{\textcolor[rgb]{0.00,0.00,0.00}{#1}}
\newcommand{\StringTok}[1]{\textcolor[rgb]{0.31,0.60,0.02}{#1}}
\newcommand{\VerbatimStringTok}[1]{\textcolor[rgb]{0.31,0.60,0.02}{#1}}
\newcommand{\SpecialStringTok}[1]{\textcolor[rgb]{0.31,0.60,0.02}{#1}}
\newcommand{\ImportTok}[1]{#1}
\newcommand{\CommentTok}[1]{\textcolor[rgb]{0.56,0.35,0.01}{\textit{#1}}}
\newcommand{\DocumentationTok}[1]{\textcolor[rgb]{0.56,0.35,0.01}{\textbf{\textit{#1}}}}
\newcommand{\AnnotationTok}[1]{\textcolor[rgb]{0.56,0.35,0.01}{\textbf{\textit{#1}}}}
\newcommand{\CommentVarTok}[1]{\textcolor[rgb]{0.56,0.35,0.01}{\textbf{\textit{#1}}}}
\newcommand{\OtherTok}[1]{\textcolor[rgb]{0.56,0.35,0.01}{#1}}
\newcommand{\FunctionTok}[1]{\textcolor[rgb]{0.00,0.00,0.00}{#1}}
\newcommand{\VariableTok}[1]{\textcolor[rgb]{0.00,0.00,0.00}{#1}}
\newcommand{\ControlFlowTok}[1]{\textcolor[rgb]{0.13,0.29,0.53}{\textbf{#1}}}
\newcommand{\OperatorTok}[1]{\textcolor[rgb]{0.81,0.36,0.00}{\textbf{#1}}}
\newcommand{\BuiltInTok}[1]{#1}
\newcommand{\ExtensionTok}[1]{#1}
\newcommand{\PreprocessorTok}[1]{\textcolor[rgb]{0.56,0.35,0.01}{\textit{#1}}}
\newcommand{\AttributeTok}[1]{\textcolor[rgb]{0.77,0.63,0.00}{#1}}
\newcommand{\RegionMarkerTok}[1]{#1}
\newcommand{\InformationTok}[1]{\textcolor[rgb]{0.56,0.35,0.01}{\textbf{\textit{#1}}}}
\newcommand{\WarningTok}[1]{\textcolor[rgb]{0.56,0.35,0.01}{\textbf{\textit{#1}}}}
\newcommand{\AlertTok}[1]{\textcolor[rgb]{0.94,0.16,0.16}{#1}}
\newcommand{\ErrorTok}[1]{\textcolor[rgb]{0.64,0.00,0.00}{\textbf{#1}}}
\newcommand{\NormalTok}[1]{#1}
\usepackage{graphicx,grffile}
\makeatletter
\def\maxwidth{\ifdim\Gin@nat@width>\linewidth\linewidth\else\Gin@nat@width\fi}
\def\maxheight{\ifdim\Gin@nat@height>\textheight\textheight\else\Gin@nat@height\fi}
\makeatother
% Scale images if necessary, so that they will not overflow the page
% margins by default, and it is still possible to overwrite the defaults
% using explicit options in \includegraphics[width, height, ...]{}
\setkeys{Gin}{width=\maxwidth,height=\maxheight,keepaspectratio}
\IfFileExists{parskip.sty}{%
\usepackage{parskip}
}{% else
\setlength{\parindent}{0pt}
\setlength{\parskip}{6pt plus 2pt minus 1pt}
}
\setlength{\emergencystretch}{3em}  % prevent overfull lines
\providecommand{\tightlist}{%
  \setlength{\itemsep}{0pt}\setlength{\parskip}{0pt}}
\setcounter{secnumdepth}{0}
% Redefines (sub)paragraphs to behave more like sections
\ifx\paragraph\undefined\else
\let\oldparagraph\paragraph
\renewcommand{\paragraph}[1]{\oldparagraph{#1}\mbox{}}
\fi
\ifx\subparagraph\undefined\else
\let\oldsubparagraph\subparagraph
\renewcommand{\subparagraph}[1]{\oldsubparagraph{#1}\mbox{}}
\fi

%%% Use protect on footnotes to avoid problems with footnotes in titles
\let\rmarkdownfootnote\footnote%
\def\footnote{\protect\rmarkdownfootnote}

%%% Change title format to be more compact
\usepackage{titling}

% Create subtitle command for use in maketitle
\providecommand{\subtitle}[1]{
  \posttitle{
    \begin{center}\large#1\end{center}
    }
}

\setlength{\droptitle}{-2em}

  \title{Tables for Open Computer Science}
    \pretitle{\vspace{\droptitle}\centering\huge}
  \posttitle{\par}
    \author{Franziska Hirt}
    \preauthor{\centering\large\emph}
  \postauthor{\par}
      \predate{\centering\large\emph}
  \postdate{\par}
    \date{13 Februar 2019}

\usepackage{booktabs}
\usepackage{longtable}
\usepackage{array}
\usepackage{multirow}
\usepackage{wrapfig}
\usepackage{float}
\usepackage{colortbl}
\usepackage{pdflscape}
\usepackage{tabu}
\usepackage{threeparttable}
\usepackage{threeparttablex}
\usepackage[normalem]{ulem}
\usepackage{makecell}
\usepackage{xcolor}

\begin{document}
\maketitle

\section{load packages and data}\label{load-packages-and-data}

Table 1:

\begin{Shaded}
\begin{Highlighting}[]
\NormalTok{table1 <-}\StringTok{ }\NormalTok{df1 }\OperatorTok
\StringTok{  }\KeywordTok{mutate_all}\NormalTok{(linebreak) }\OperatorTok
\StringTok{  }\KeywordTok{kable}\NormalTok{(}\DataTypeTok{align=}\KeywordTok{c}\NormalTok{(}\KeywordTok{rep}\NormalTok{(}\StringTok{"l"}\NormalTok{,}\DecValTok{5}\NormalTok{)), }\DataTypeTok{booktabs =}\NormalTok{ T, }\DataTypeTok{escape =}\NormalTok{ F, }\DataTypeTok{caption =} \StringTok{"Descriptives of FaceReader's estimates (aggregated as mean and mean of peak values) and students' self-reports"}\NormalTok{, }\DataTypeTok{col.names =} \KeywordTok{c}\NormalTok{(}\StringTok{" "}\NormalTok{, }\StringTok{"Self-report after"}\NormalTok{, }\StringTok{"FaceReader mean"}\NormalTok{, }\StringTok{"FaceReader mean of peak values"}\NormalTok{)) }\OperatorTok
\StringTok{  }\KeywordTok{column_spec}\NormalTok{(}\DecValTok{2}\OperatorTok{:}\DecValTok{4}\NormalTok{, }\DataTypeTok{width =} \StringTok{"1.8cm"}\NormalTok{) }\OperatorTok
\StringTok{  }\KeywordTok{column_spec}\NormalTok{(}\DecValTok{1}\NormalTok{, }\DataTypeTok{width =} \StringTok{"1.2cm"}\NormalTok{) }\OperatorTok\StringTok{ }
\StringTok{  }\KeywordTok{collapse_rows}\NormalTok{(}\DataTypeTok{columns =} \DecValTok{1}\OperatorTok{:}\DecValTok{3}\NormalTok{, }\DataTypeTok{latex_hline =} \StringTok{"full"}\NormalTok{) }\OperatorTok
\StringTok{  }\KeywordTok{footnote}\NormalTok{(}\DataTypeTok{general=} \StringTok{"The table presents the mean (SD, Scale range -- higher values indicating higher intensity)"}\NormalTok{,}
           \DataTypeTok{title_format =} \KeywordTok{c}\NormalTok{(}\StringTok{"italic"}\NormalTok{), }\DataTypeTok{threeparttable =}\NormalTok{ T) }\CommentTok{#so that captation is wrapped }
\NormalTok{table1}
\end{Highlighting}
\end{Shaded}

\begin{table}[t]

\caption{\label{tab:unnamed-chunk-2}Descriptives of FaceReader's estimates (aggregated as mean and mean of peak values) and students' self-reports}
\centering
\begin{threeparttable}
\begin{tabular}{>{\raggedright\arraybackslash}p{1.2cm}>{\raggedright\arraybackslash}p{1.8cm}>{\raggedright\arraybackslash}p{1.8cm}>{\raggedright\arraybackslash}p{1.8cm}}
\toprule
  & Self-report after & FaceReader mean & FaceReader mean of peak values\\
\midrule
Interest & 3.8 \newline (0.93; 1-5) & 0.01 \newline (0.04; 0-1) & 0.03 \newline (0.08; 0-1)\\
\cmidrule{1-4}
Boredom & 1.44 \newline (0.79; 1-5) & 0.06 \newline (0.15; 0-1) & 0.32 \newline (0.28; 0-1)\\
\cmidrule{1-4}
 &  &  & pos: 0.14 \newline (0.18, 0 -1)\\
\cmidrule{4-4}
\multirow{-2}{1.2cm}{\raggedright\arraybackslash Valence} & \multirow{-2}{1.8cm}{\raggedright\arraybackslash 6.94 \newline (1.13; 1-5)} & \multirow{-2}{1.8cm}{\raggedright\arraybackslash -0.12 \newline (0.15; 0-5)} & neg: -0.28 \newline (0.21, -1-0)\\
\bottomrule
\end{tabular}
\begin{tablenotes}
\item \textit{Note: } 
\item The table presents the mean (SD, Scale range -- higher values indicating higher intensity)
\end{tablenotes}
\end{threeparttable}
\end{table}

Table 2: \#\# Usually it is recommended to use column\_spec before
collapse\_rows, especially in LaTeX, to get a desired result.

\section{\texorpdfstring{df{[}1,5{]} \textless{}-
cell\_spec(df{[}1,5{]}, ``latex'', underline =
T)}{df{[}1,5{]} \textless{}- cell\_spec(df{[}1,5{]}, latex, underline = T)}}\label{df15---cell_specdf15-latex-underline-t}

\section{\texorpdfstring{df2X{[}1,2{]} \textless{}-
paste0(``\textbackslash{}underline\{'', df{[}1,2{]},
``\}'')}{df2X{[}1,2{]} \textless{}- paste0(\textbackslash{}underline\{, df{[}1,2{]}, \})}}\label{df2x12---paste0underline-df12}

\section{color specific cells (those of the null
model)}\label{color-specific-cells-those-of-the-null-model}

\section{each row has to be done seperately
:(}\label{each-row-has-to-be-done-seperately}

df2x{[}1, c(2,3,4){]} \textless{}- cell\_spec(df2x{[}1, c(2,3,4){]},
background = ``\#BBBBBB'') df2x{[}5, c(2,3,4){]} \textless{}-
cell\_spec(df2x{[}5, c(2,3,4){]}, background = ``\#BBBBBB'') df2x{[}9,
c(2,3,4){]} \textless{}- cell\_spec(df2x{[}9, c(2,3,4){]}, background =
``\#BBBBBB'')

\#row\_spec(1, background = ``\#BBBBBB'')\%\textgreater{}\%
\#row\_spec(6, background = ``\#BBBBBB'')\%\textgreater{}\%
\#row\_spec(11, background = ``\#BBBBBB'')\%\textgreater{}\%

\begin{Shaded}
\begin{Highlighting}[]
\CommentTok{# rotate first column (named "x")}
\NormalTok{df2x <-}\StringTok{ }\NormalTok{df2 }\OperatorTok\StringTok{ }\KeywordTok{mutate}\NormalTok{(}\DataTypeTok{x =} \KeywordTok{text_spec}\NormalTok{(x, }\StringTok{"latex"}\NormalTok{, }\DataTypeTok{angle =} \DecValTok{90}\NormalTok{)) }\CommentTok{#, align = "c"}


\CommentTok{# build table  }
\NormalTok{table2 <-}\StringTok{ }\NormalTok{df2x }\OperatorTok
\StringTok{  }\KeywordTok{kable}\NormalTok{(}\DataTypeTok{align=}\KeywordTok{c}\NormalTok{(}\StringTok{"c"}\NormalTok{, }\KeywordTok{rep}\NormalTok{(}\StringTok{"l"}\NormalTok{,}\DecValTok{5}\NormalTok{)), }\DataTypeTok{escape =}\NormalTok{ F, }\DataTypeTok{booktabs =}\NormalTok{ T, }\DataTypeTok{caption =} \StringTok{"Overview of the regression coefficients"}\NormalTok{, }\DataTypeTok{col.names =} \KeywordTok{c}\NormalTok{(}\StringTok{" "}\NormalTok{, }\StringTok{"Model with aggregation method for FaceReader's estimates"}\NormalTok{, }\StringTok{"Regression coefficient and its standard error"}\NormalTok{, }\StringTok{"Credible interval"}\NormalTok{, }\StringTok{"Number of observations"}\NormalTok{)) }\OperatorTok
\StringTok{  }\KeywordTok{column_spec}\NormalTok{(}\DecValTok{1}\NormalTok{, }\DataTypeTok{width=}\StringTok{"0.8cm"}\NormalTok{) }\OperatorTok\StringTok{ }\CommentTok{# background = "white" not working in latex}
\StringTok{  }\KeywordTok{column_spec}\NormalTok{(}\DecValTok{2}\NormalTok{, }\DataTypeTok{width=}\StringTok{"6.5cm"}\NormalTok{) }\OperatorTok
\StringTok{  }\KeywordTok{column_spec}\NormalTok{(}\DecValTok{3}\NormalTok{, }\DataTypeTok{width=}\StringTok{"4.5cm"}\NormalTok{) }\OperatorTok
\StringTok{  }\KeywordTok{column_spec}\NormalTok{(}\DecValTok{4}\NormalTok{, }\DataTypeTok{width=}\StringTok{"3cm"}\NormalTok{) }\OperatorTok
\StringTok{  }\KeywordTok{column_spec}\NormalTok{(}\DecValTok{5}\NormalTok{, }\DataTypeTok{width=}\StringTok{"1.4cm"}\NormalTok{) }\OperatorTok
\StringTok{  }\KeywordTok{collapse_rows}\NormalTok{(}\DataTypeTok{columns =} \KeywordTok{c}\NormalTok{(}\DecValTok{1}\NormalTok{,}\DecValTok{2}\NormalTok{,}\DecValTok{5}\NormalTok{), }\DataTypeTok{valign =} \StringTok{"middle"}\NormalTok{, }\DataTypeTok{latex_hline =} \StringTok{"full"}\NormalTok{) }\OperatorTok
\StringTok{  }\KeywordTok{kable_styling}\NormalTok{(}\DataTypeTok{full_width =}\NormalTok{ F, }\DataTypeTok{protect_latex =}\NormalTok{ T) }\OperatorTok\StringTok{ }\CommentTok{# LaTeX code between dollar protected from HTML escaping}
\StringTok{  }\KeywordTok{footnote}\NormalTok{(}\DataTypeTok{general=}\StringTok{"Coefficients are based on standardized predictors (FaceReader), but unstandardized outcomes (self-reports). Some missing self-reports reduced the sample size of specific analyses."}\NormalTok{,}
           \DataTypeTok{title_format =} \KeywordTok{c}\NormalTok{(}\StringTok{"italic"}\NormalTok{), }\DataTypeTok{threeparttable =}\NormalTok{ T) }\CommentTok{# threeparttable for layout included (! does not work with full_width = T, unless longtable = T, but longtable not possible in two column layout)}
\NormalTok{table2}
\end{Highlighting}
\end{Shaded}

\begin{table}[t]

\caption{\label{tab:unnamed-chunk-3}Overview of the regression coefficients}
\centering
\begin{threeparttable}
\begin{tabular}{>{\centering\arraybackslash}p{0.8cm}>{\raggedright\arraybackslash}p{6.5cm}>{\raggedright\arraybackslash}p{4.5cm}>{\raggedright\arraybackslash}p{3cm}>{\raggedright\arraybackslash}p{1.4cm}}
\toprule
  & Model with aggregation method for FaceReader's estimates & Regression coefficient and its standard error & Credible interval & Number of observations\\
\midrule
 & mean as predictor & $\textit{b}  = -0.17, \textit{SE}  = 0.28$ & CrI = [-0.71, 0.38] & \\
\cmidrule{2-4}
 &  & $\textit{b\_mean*sd}  = -0.05 , \textit{SE}  = 0.30$ & CrI = [-0.60 ,  0.58] & \\
\cmidrule{3-4}
 &  & $\textit{b\_mean}  = -0.35 , \textit{SE}  = 1.00$ & CrI = [-2.38,  1.59] & \\
\cmidrule{3-4}
 & \multirow{-3}{6.5cm}{\raggedright\arraybackslash interaction of mean and SD of mean as predictor} & $\textit{b\_sd}  = 0.21, \textit{SE}  = 0.34$ & CrI = [-0.44, 0.88] & \\
\cmidrule{2-4}
\multirow{-5}{0.8cm}{\centering\arraybackslash \rotatebox{90}{Interest}} & mean over the 10\% of the highest values as predictor & $\textit{b}  = -0.07, \textit{SE}  = 0.19$ & CrI = [-0.44, 0.29] & \multirow{-5}{1.4cm}{\raggedright\arraybackslash \textit{obs} = 203}\\
\cmidrule{1-5}
 & mean as predictor & $\textit{b}  = -0.10, \textit{SE}  = 0.18$ & CrI = [-0.47, 0.23] & \\
\cmidrule{2-4}
 &  & $\textit{b\_mean*sd}  = 0.38, \textit{SE}  = 0.24$ & CrI = [-0.03, 0.90] & \\
\cmidrule{3-4}
 &  & \textit{b\_mean} = -0.25, \textit{SE}  = 0.35 & CrI = [-1.02, 0.35] & \\
\cmidrule{3-4}
 & \multirow{-3}{6.5cm}{\raggedright\arraybackslash interaction of mean and SD of mean as predictor} & $\textit{b\_sd}  = -0.03, \textit{SE}  = 0.30$ & CrI = [-0.61, 0.58] & \\
\cmidrule{2-4}
\multirow{-5}{0.8cm}{\centering\arraybackslash \rotatebox{90}{Boredom}} & mean over the 10\% of the highest values as predictor & $\textit{b}  = -0.16, \textit{SE}  = 0.18$ & CrI = [-0.53, 0.16 ] & \multirow{-5}{1.4cm}{\raggedright\arraybackslash \textit{obs} = 204}\\
\cmidrule{1-5}
 & mean as predictor & $\textit{b}  = -0.01, \textit{SE}  = 0.15$ & CrI = [-0.30, 0.28] & \\
\cmidrule{2-4}
 &  & $\textit{b\_mean*sd}  = -0.13, \textit{SE}  = 0.15$ & CrI = [-0.43, 0.16] & \\
\cmidrule{3-4}
 &  & $\textit{b\_mean}  = 0.16, \textit{SE}  = 0.24$ & CrI = [-0.31, 0.63] & \\
\cmidrule{3-4}
 & \multirow{-3}{6.5cm}{\raggedright\arraybackslash interaction of mean and SD of mean as predictor} & $\textit{b\_sd}  = 0.03, \textit{SE}  = 0.14$ & CrI = [-0.23, 0.30] & \\
\cmidrule{2-4}
 &  & $\textit{b\_pos}  = 0.04, \textit{SE}  = 0.12$ & CrI = [-0.20, 0.28] & \\
\cmidrule{3-4}
\multirow{-6}{0.8cm}{\centering\arraybackslash \rotatebox{90}{Valence}} & \multirow{-2}{6.5cm}{\raggedright\arraybackslash mean over the 10\% of the most extreme positive and  negative values as two separate predictors} & $\textit{b\_neg}  = -0.05, \textit{SE}  = 0.14$ & CrI = [-0.32, 0.23] & \multirow{-6}{1.4cm}{\raggedright\arraybackslash \textit{obs} = 193}\\
\bottomrule
\end{tabular}
\begin{tablenotes}
\item \textit{Note: } 
\item Coefficients are based on standardized predictors (FaceReader), but unstandardized outcomes (self-reports). Some missing self-reports reduced the sample size of specific analyses.
\end{tablenotes}
\end{threeparttable}
\end{table}


\end{document}
